\documentclass[utf8]{article}

\usepackage[utf8]{inputenc}

\usepackage[parfill]{parskip}

\usepackage{amsmath}
\usepackage{amssymb}
\usepackage{amsfonts}
\usepackage{graphicx}
\usepackage{float}
\usepackage{algorithm}
\usepackage{algorithmicx}
\usepackage{algpseudocode}
\usepackage{fullpage}
\usepackage{multirow}

% -----------------------------------------------------


\title{Hashtables}
\author{Bourgeois Noé}
\date{16 mai 2021}

\begin{document}
\maketitle
\tableofcontents

\newpage

% -----------------------------------------------------

\section{Introduction}
 Il existe différentes méthodes de résolution des collisions des tables de hachage.
 Pour étudier les caractéristiques de 3 d'entre elles, nous avons implémenté une classe pour chacune, représentant un dictionnaire dont les clefs (uniques comme dans un dict en Python, ce qui implique que lors d’une insertion à une clef k donnée dans le conteneur, si cette clef existe déjà, il faut remplacer la valeur stockée) et valeurs sont des chaînes de caractères (donc des objets de type str):

\begin{tabular}{|c|c|c|c|} 
\hline
classe & DictOpenAddressing & DictChainingLinkedList & DictChainingSkipList \\
\hline

gestion des collisions & double hachage & liste chaînée & skiplist \\
\hline

\multirow{2}{*}{algorithme(s) de hachage} & \multicolumn{3}{c|}{djb2} \\\cline{2-4}
                                        & KR & \multicolumn{2}{|c|}{} \\
\hline

\multirow{8}{*}{methodes} & \multicolumn{3}{l|}{\_\_init\_\_(self, m):} \\
                            & \multicolumn{3}{l|}{initialise l’objet} \\\cline{2-4}
                        & \multicolumn{3}{l|}{\_\_len\_\_(self):} \\
                         & \multicolumn{3}{l|}{retourne le nombre d’éléments stockés (et non la taille de la table)} \\\cline{2-4}
                         
                         & \multicolumn{3}{l|}{insert(self, key, value):}\\
                         & \multicolumn{3}{l|}{insère l’élément value associé à la clef key }\\
                         & \multicolumn{3}{l|}{lance une OverflowError
                            si l’insertion est impossible)} \\\cline{2-4}
                            
                         & \multicolumn{3}{l|}{search(self, key):}\\
                         & \multicolumn{3}{l|}{renvoie la valeur associée à la clef key si cette dernière existe dans le
                        conteneur}\\
                        & \multicolumn{3}{l|}{lance une KeyError sinon} \\\cline{2-4}
                         & \multicolumn{3}{l|}{\_\_setitem\_\_(self, key, value):}\\
                         & \multicolumn{3}{l|}{appelle insert} \\\cline{2-4}
                         
                         & \multicolumn{3}{l|}{\_\_getitem\_\_(self, key):}\\
                         & \multicolumn{3}{l|}{appelle search}\\\cline{2-4}
                            
                         & \multicolumn{3}{l|}{delete(self, key):}\\
                         & \multicolumn{3}{l|}{retire l’élément de clef key du conteneur s’il existe et lance une KeyError sinon} \\\cline{2-4}
                         
                         & \multicolumn{3}{l|}{\_\_delitem\_\_(self, key):}\\
                         & \multicolumn{3}{l|}{ appelle delete} \\\cline{2-4}
 \hline
 propriété & \multicolumn{3}{l|}{load\_factor:} \\
            & \multicolumn{3}{l|}{renvoie le facteur de charge:}\\ 
            & \multicolumn{3}{l|}{nombre d’éléments
    stockés dans le conteneur/taille de la table} \\

\hline

\end{tabular}

\section{Méthodes}
\subsection{KR}

% Exemple d'ajout d'un pseudo-code avec les packages algorithm/algorithmicx
\begin{algorithm}
\caption{Algorithme de Kernighan \& Ritchie}
\begin{algorithmic}[1]
\Procedure{KR}{\textbf{string} $s$}
\State \verb+hash+ $\gets$ 0
\ForAll{\textbf{char} $c$ \textbf{in} $s$}
  \State \verb+hash+ $\gets$ \verb+hash+ + $c$
\EndFor
\State \Return \verb+hash+
\EndProcedure
\end{algorithmic}
\end{algorithm}

\subsection{djb2}

\begin{algorithm}
\caption{Algorithme djb2 de Daniel J. Bernstein}
\begin{algorithmic}[1]
\Procedure{djb2}{\textbf{string} $s$}
\State \verb+hash+ $\gets$ 0x1505
\ForAll{\textbf{char} $c$ \textbf{in} $s$}
  \State \verb+hash+ $\gets$ 33x \verb+hash+ + $c$
\EndFor
\State \Return \verb+hash+
\EndProcedure
\end{algorithmic}
\end{algorithm}

\subsubsection{Sous-sous-section}

\newpage

\section{Résultats}
    \begin{figure}[H]
  \centering
	\includegraphics[scale=0.4]{img/logo.png}
  \label{fig:logo}
\end{figure}

     statistiques sur le temps nécessaire pour l’exécution des méthodes insert, search et delete de
la classe DictOpenAddressing pour différentes valeurs de α ainsi que des représentations graphiques

    comparaison du nombre de sondages effectués par les méthodes insert, search et delete des
classes DictChainingLinkedList et DictChainingSkipList ;
    
    discussion sur les avantages et inconv´enients de ces deux classes
% Exemple d'ajout d'un tableau
\begin{center}
\begin{tabular}{|c|c|}
\hline
1$^{\text{ère}}$ ligne & test \\
2$^{\text{ème}}$ ligne & test 2 \\
 
\hline
\end{tabular}
\end{center}
comparaison de ces valeurs ainsi que du nombre moyen de sondages avant de trouver une cellule
libre avec les r´esultats th´eoriques vus au cours

\section{Discussion}
l’insertion par chaînage ne peut pas lancer d’exception car l’insertion y est toujours possible
    discussion sur l’hypoth`ese de hachage uniforme pour les diff´erentes fonctions de hachage impl´ement´ees
\section{Conclusion}
    L'adressage ouvert est plus performant que le chaînage mais est limité par la taille de son dictionnaire.

\end{document}