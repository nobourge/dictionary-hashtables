\documentclass[utf8]{article}

\usepackage[utf8]{inputenc}

\usepackage[parfill]{parskip}

\usepackage{amsmath}
\usepackage{amssymb}
\usepackage{amsfonts}
\usepackage{graphicx}
\usepackage{float}
\usepackage{algorithm}
\usepackage{algorithmicx}
\usepackage{algpseudocode}
\usepackage{fullpage}

% -----------------------------------------------------


\title{Titre}
\author{Nom prénom}
\date{16 mai 2021}

\begin{document}
\maketitle
\tableofcontents

\newpage

% -----------------------------------------------------

\section{Introduction}

% Insérer l'introduction ici

% Exemple d'ajout d'une image
\begin{figure}[H]
  \centering
	\includegraphics[scale=0.4]{img/logo.png}
  \label{fig:logo}
\end{figure}

\section{Méthodes}

% Exemple d'ajout d'un pseudo-code avec les packages algorithm/algorithmicx
\begin{algorithm}
\caption{Algorithme de Kernighan \& Ritchie}
\begin{algorithmic}[1]
\Procedure{KR}{\textbf{string} $s$}
\State \verb+hash+ $\gets$ 0
\ForAll{\textbf{char} $c$ \textbf{in} $s$}
  \State \verb+hash+ $\gets$ \verb+hash+ + $c$
\EndFor
\State \Return \verb+hash+
\EndProcedure
\end{algorithmic}
\end{algorithm}

\subsection{Sous-section}

\subsubsection{Sous-sous-section}

\section{Résultats}

% Exemple d'ajout d'un tableau
\begin{center}
\begin{tabular}{|c|c|}
\hline
1$^{\text{ère}}$ ligne & test \\
2$^{\text{ème}}$ ligne & test 2 \\
\hline
\end{tabular}
\end{center}

\section{Discussion}

\section{Conclusion}


\end{document}
